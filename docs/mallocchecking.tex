For Malloc checking, we suppose that a security program should always check the
return value of \texttt{malloc}. This is a strong yet reasonable requirement
for ensuring secure code, because if the pointer fails to be assigned a value
due to a lack of space for allocation, an attacker may be able to
exploit a NULL pointer dereference at a later point in the program.

To demonstrate how our plugin check works, observe the example shown in
\ref{fig:malloc}. In this example, the function \texttt{malloc} is called at
address \texttt{0x82d0}. Followed by the instruction, we have a comparison
instruction which compare the return value \texttt{r0} to constant value
\texttt{0}. This piece of code is thus a candidate for \texttt{malloc}
checking.

Note that the comparison instruction is not necessarily available immediately after
the malloc call instruction. Also, a more complicated case could be that the
return value is passed to another register which in turn is checked
instead of \texttt{r0}.

Another example that we came across is from binary coreutils\_O1\_[, which is
shown as following:

\begin{center}
\lstset{language=C,
caption=Malloc disassembly, breaklines=true, basicstyle=\tiny, numbers=none}
\begin{lstlisting}
    d4d0:       e0800009        add     r0, r0, r9
    d4d4:       e280a002        add     sl, r0, 2
    d4d8:       e2800003        add     r0, r0, 3
    d4dc:       ebffee80        bl      8ee4 <malloc@plt>             *
    d4e0:       e1a08000        mov     r8, r0                        *
    d4e4:       ea000006        b       d504 <locale_charset+0x208>
    d4e8:       e0891000        add     r1, r9, r0
    d4ec:       e081100a        add     r1, r1, sl
    d4f0:       e281a002        add     sl, r1, 2
    d4f4:       e1a00004        mov     r0, r4
    d4f8:       e2811003        add     r1, r1, 3
    d4fc:       ebffee4b        bl      8e30 <realloc@plt>
    d500:       e1a08000        mov     r8, r0
    d504:       e3580000        cmp     r8, 0                        *
    d508:       1a000005        bne     d524 <locale_charset+0x228>   *
    d50c:       e1a00004        mov     r0, r4
\end{lstlisting}
\end{center}
In this example, after calling malloc at address d4dc, the program copies the
return value to register \texttt{r8}, and then to address \texttt{d504} where
it compares the value of \texttt{r8} to constant 0. The reason of compiling the
program in such way is mainly because of optimization. Specifically, this
helps decreasing the redundant code, by reusing \texttt{d504: cmp r8, 0} and
the subsequent instructions.

To address this problem, we use the forward-data-deps plugin to help finding the
instructions that depends on the return value. Here dependence means that one or
more variables used in the instruction depends on register \texttt{r0}. For
example, given instruction \texttt{d4e0: mov r8, r0}, we find that instruction
\texttt{d504: cmp r8, 0} depends on \texttt{r0}, because the value of
\texttt{r8} is from \texttt{r0}. Also, instruction d508 depends on r0, because
the flag registers are determined by
the \texttt{d504}, and in \texttt{d504 r8} depends on \texttt{r0}.

Looking back to the example, and using forward-data-deps plugin, we get all
dependence instructions once we find a move instruction with \texttt{r0} as
source. For In the next step, we check if there is \texttt{cmp} instruction
among all instructions we found. If there is one, we consider call is safe. If
not, we consider the call unsafe.
