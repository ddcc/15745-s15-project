For Malloc checking, we suppose that a security program should always check the
return value of \texttt{malloc}. This is a strong yet reasonable requirement for
security coding, because if the point is failed to be assigned and the space for
the pointer is failed to be allocated, attack may use this failed pointer to
exploit the program.

To demonstrate how our work runs, we use the example shown in \ref{fig:malloc}.
In this example, function \texttt{malloc} is called at address \texttt{0x82d0}.
Followed by the instruction, we have a comparison instruction which compare the
return value \texttt{r0} to constant value \texttt{0}. This piece of code is
thus a security one for \texttt{malloc} checking.

Note that the comparison instruction is not necessarily to be immediately after
the malloc call instruction. Also, a more complicated case could be that, the
return value is passed to other register and then the other register is checked
instead of `r0`.

Another example that we come across is from binary coreutils\_O1\_[, which is
shown as following:

\begin{center}
\lstset{language=C,
caption=Malloc disassembly, breaklines=true, basicstyle=\tiny, numbers=none}
\begin{lstlisting}
    d4d0:       e0800009        add     r0, r0, r9
    d4d4:       e280a002        add     sl, r0, 2
    d4d8:       e2800003        add     r0, r0, 3
    d4dc:       ebffee80        bl      8ee4 <malloc@plt>             *
    d4e0:       e1a08000        mov     r8, r0                        *
    d4e4:       ea000006        b       d504 <locale_charset+0x208>
    d4e8:       e0891000        add     r1, r9, r0
    d4ec:       e081100a        add     r1, r1, sl
    d4f0:       e281a002        add     sl, r1, 2
    d4f4:       e1a00004        mov     r0, r4
    d4f8:       e2811003        add     r1, r1, 3
    d4fc:       ebffee4b        bl      8e30 <realloc@plt>
    d500:       e1a08000        mov     r8, r0
    d504:       e3580000        cmp     r8, 0                        *
    d508:       1a000005        bne     d524 <locale_charset+0x228>   *
    d50c:       e1a00004        mov     r0, r4
\end{lstlisting}
\end{center}
In this example, after calling malloc at address d4dc, the program copies the
return value to register r8, and then just tp address d504 where it compares the
value of r8 to constant 0. The reason of compiling the program in such way is
mainly because of optimization. In specifically, this helps decreasing the
redundant code, by reusing \texttt{d504: cmp r8, 0} and the following instructions.

To addres this problem, we use the forward-data-deps plugin to help finding the
instructions that depends on the return value. Here dependence means that one or
more variables used in the instruction depends on register r0. For example,
given instruction \texttt{d4e0: mov r8, r0}, we find that instruction d504: cmp r8, 0
depends on r0, because the value of r8 is from r0. Also, instruction d508
depends on r0, because the flag registers are determined by the d504, and in
d504 r8 depends on r0.

Now back to the example, using forward-data-deps plugin, we get all dependence
instructions once we find a move instruction with r0 as source. For the next
step, we check if there is cmp instruction among all instructions we found. If
there is one, we consider the call is safe. If not, we consider the call is
unsafe.
