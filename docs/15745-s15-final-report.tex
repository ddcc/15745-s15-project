\documentclass[letterpaper,11pt]{article}
\usepackage[margin=1in]{geometry}
\usepackage[hyphens]{url}
\usepackage{colortbl}
\usepackage{listings}
\begin{document}

\date{\today}

\title{\Large \bf Memory Safety Bug Hunting with BAP} 

\author{Tiffany Bao, Dominic Chen, Rijnard van Tonder}

\maketitle

\section{Introduction}
\label{intro}

\paragraph{The Problem}

Although numerous debugging tools have been developed for performing static and
dynamic software analysis, the majority of these require access to source
code or utilization of special compile­time flags to embed debug information
within the output binary. In contrast, little effort has been spent on
developing static binary analysis tools which can directly analyze
off­the­shelf executables without executing them. For this project, we plan to
focus on a subset of common programming bugs that tend to introduce security
vulnerabilities, known as memory­safety errors. Examples of these include
buffer overflows, double frees, and uninitialized variables. In particular, we
define \emph{Security Outcomes} with respect to certain memory-safety errors,
and subsequently implement and test static analyses that yield these outcomes.
All analyses were written against our group's Binary Analysis Platform
(BAP) \cite{bap, brumley2011bap}.

A factor of this research area that appeals to many is the ability to
conclude whether vulnerabilities do or do not exist by performing
an analysis purely on the binary level. Due to the inherent difficulties
of operating at this level, %we can expand on what these difficulties are...
our approach to the problem starts out coarse, and continues to be
incrementally refined as we become more familiar with the numerous
barriers that inhibit vulnerability detection in binaries. In this respect,
we have much to say towards lessons learned in \S\ref{lessonslearned}.

\subsection{Approach Overview}

At the outset of this project, we planned to implement and test the static
analyses with respect to our original proposal:

\begin{enumerate}
  \item Intraprocedural data dependency resolution of instructions
  \item Intraprocedural detection of unchecked memory allocations
  \item Basic intraprocedural detection of buffer overflows in the stack
  \item Basic intraprocedural detection of buffer overflows in the heap
\end{enumerate}

Equipped with the knowledge of data flow dependency between instructions (1),
we determined to reason about various properties of typical vulnerabilities
that we could check (2-4). For example, we might like to follow how arguments
to a `strcpy` call are influenced by previous instructions as a function of
the stack register.

\subsection{Related Work}

Engler et. al. found numerous bugs with system-specific static analyses,
implemented by way of a meta-compilation language for the \texttt{gcc} compiler
\cite{dawson,dawson2}. Analyses are composed in the form of correctness rules
and automatic inference of rules form source code. Memory errors that are
checked include null pointer bugs, not checking allocation results, and uses of
freed pointers. This relates to our goal in that we would ideally like to perform
similar checks, albeit at the level of complexity of binary code. Whereas
Engler's work matured into a generic platform for specifying analyses, we must
start first with an ad-hoc approach to gain insight and familiarity in the
problem domain presented by binaries.

% needs more

\subsection{Contributions}

Our key technical contributions are as follows:

\begin{enumerate}
  \item A dataflow analysis framework for BAP.
  \item General plugins for BAP that deliver
    \begin{enumerate}
      \item reaching definitions and
      \item data flow dependency resolution (use-def and def-use chains).
    \end{enumerate}
  \item Security-specific plugins for BAP that produce security outcomes by checking
    \begin{enumerate}
      \item whether memory allocation checks are performed on `memcpy`,
      \item whether calls such as strcpy are safe from buffer overflows under
        specific conditions, and
      \item whether malloc calls... % dominic?
    \end{enumerate}
  \item Results of testing the aforementioned plugins on the \texttt{GNU CoreUtils}
    software suite.
\end{enumerate}

\section{Approach}

Our approach toward ensuring security properties.

Initially, we determined to reason about each of the cases above that relate
to vulnerabilities (2-4). For example, if we identified the following call
in a binary:

\begin{center}
\lstset{language=C, label=strcpy,
caption=strcpy-call, breaklines=true, basicstyle=\tiny, numbers=left}
\begin{lstlisting}
strcpy(src,dst)
\end{lstlisting}
\end{center}

We would like to 
Need basic data dependencies.


\subsection{Unchecked Memory}

\subsection{Stack-based Security Outcomes}

% heap-based? or malloc?
\subsection{Malloc-based Security Outcomes}






\section{Experimental Setup}

\subsection{}


\section{Experimental Evaluation}



\section{Surprises and Lessons Learned}
\label{lessonslearned}



\section{Conclusions and Future Work}



\clearpage
\section{Related Work}

\nocite{*}
{\bibliographystyle{acm}
\bibliography{b1}}

\end{document}
